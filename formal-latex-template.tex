\documentclass[12pt]{article}

% -------------------- Packages -------------------- %

% Font/input basics
\usepackage[T1]{fontenc}
\usepackage[utf8]{inputenc}
\usepackage{lmodern}

% Critical math packages
\usepackage{amsmath}
\usepackage{amssymb}
\usepackage{mathtools}

% Tools for theorems, including restatement capability
\usepackage{amsthm}
\usepackage{thmtools}
\usepackage{thm-restate}

% Local table of contents capability
\usepackage{etoc}

% General formatting
\usepackage[margin=1in]{geometry}
\usepackage{setspace}
\usepackage{booktabs}
\usepackage[labelsep=period,font=small,labelfont=bf,margin=5ex]{caption}

% Bibliography and references
\usepackage{multibib}
\usepackage[sort]{natbib}
\usepackage[bookmarks=false]{hyperref}

% -------------------- Setup -------------------- %

% Don't highlight references/links
\hypersetup{
  colorlinks=true,
  citecolor=black,
  linkcolor=black,
  urlcolor=black
}

% Set up natbib
\bibpunct{(}{)}{;}{a}{}{,}  % citations like (Fearon 1995; Powell 1996, 1999)
\def\citeapos#1{\citeauthor{#1}'s (\citeyear{#1})}  % Sartori's (2002)
\renewcommand{\harvardurl}[1]{\textbf{URL:} \url{#1}}  % less-bad URLs in bibliography
\newcites{app}{Additional References}

% Theorem environments (add more if needed)
\declaretheorem{lemma}
\declaretheorem{corollary}
\declaretheorem{proposition}

% -------------------- Metadata -------------------- %

\title{A \LaTeX\ Template for Formal Papers in Political Science%
  \thanks{I thank 14 years of accumulated Google and Stack Overflow searches for directing me to the various packages I needed to make this template.}
}
\author{Brenton Kenkel%
  \thanks{Assistant Professor, Department of Political Science, Vanderbilt University. \url{brenton.kenkel@vanderbilt.edu}.}
  \and Fake Second Author%
  \thanks{I only put this here because it annoys me when people don't use \texttt{\textbackslash{}and} to separate author names in coauthored \LaTeX\ documents.}
}

\begin{document}

% -------------------- Title page -------------------- %

\setcounter{page}{0}
\maketitle
\thispagestyle{empty}

\begin{abstract}
  This is my abstract.
  It is 150 or fewer words, per common journal requirements.
  It succinctly states my research question and main results in plain language.
\end{abstract}

% -------------------- Paper body -------------------- %

\clearpage
\doublespacing

My introduction would go here.
I wouldn't waste precious space on an unnecessary ``Introduction'' section header.
Papers will be cited here \citep{walter1997critical}.

\section{The Math}

I might write some equations that I want to refer to later:
\begin{equation}
  \label{eq:envelope}
  \frac{d U(x^*(\alpha), \alpha)}{d \alpha} = \frac{\partial U(x^*(\alpha), \alpha)}{\partial \alpha}.
\end{equation}
I'll use \verb|\autoref| to do all my references, including when I refer to \autoref{eq:envelope}.

I'll also state some formal results whose proofs live in the Appendix.

\begin{restatable}{proposition}{resMaximizer}
  \label{res:maximizer}
  Assume the agent's objective is to choose $x$ to maximize
  \begin{displaymath}
    U(x, \alpha) = - (x + 2 \alpha)^2,
  \end{displaymath}
  where $\alpha \in \mathbb{R}$ is exogenous.
  The optimal choice is $x^* = - 2 \alpha$.
\end{restatable}

Ordinarily you would make this with code like:
\singlespacing
\begin{verbatim}
\begin{proposition}
  \label{res:maximizer}
  Assume the agent's objective is to choose $x$ to maximize ...
\end{proposition}
\end{verbatim}
\doublespacing
Instead I use the \verb|restatable| environment as follows:
\singlespacing
\begin{verbatim}
\begin{restatable}{proposition}{resMaximizer}
  \label{res:maximizer}
  Assume the agent's objective is to choose $x$ to maximize ...
\end{restatable}
\end{verbatim}
\doublespacing
The first argument to the \verb|restatable| environment tells it which theorem environment to use.
I've set up this template so the options are \verb|proposition|, \verb|lemma|, and \verb|corollary|, but it's easy to add more.
The second argument specifies the command you want to use to restate the result later on.
So when I want to reprint the exact text of the proposition, I'll use \verb|\resMaximizer*|, as you'll see below in the source code for the Appendix.
I always start my commands with \verb|res| (as in ``result'') but that's just personal style.

One last note: we can refer to the result by its full name using \verb|\autoref{res:maximizer}|, where \verb|res:maximizer| is the label I set just after declaring the environment.
This shows up as \autoref{res:maximizer}.
It's convenient to do this instead of the usual \verb|Proposition~\ref{res:maximizer}|, because then I can change propositions to lemmas (or vice versa) without having to go through and manually edit references.


\section{Tables and Figures}

\begin{table}[tbp]
  \centering
  \begin{tabular}{llc}
    \toprule
    Letter & Name & Brenton's rating \\
    \midrule
    $\alpha$ & alpha & 8.4 \\
    $\beta$ & beta & 9.1 \\
    $\chi$ & chi & 4.6 \\
    \bottomrule
  \end{tabular}
  \caption{My ratings of a few Greek letters.  I haven't included all of the Greek letters because I am lazy.  This caption is excessively long in order to show you how excessively long captions are formatted.}
  \label{tab:greek}
\end{table}

I use the \verb|caption| package to make the caption formatting less ugly than the \LaTeX\ default.
Similarly, I use \verb|booktabs| to make tables less ugly.
You can see the results in \autoref{tab:greek}.
As you should have come to expect by now, I made that reference via \verb|\autoref{tab:greek}|, not via the uglier and less-robust \verb|Table~\ref{tab:greek}|.


\singlespacing
\bibliography{references}
\bibliographystyle{apsr}


% -------------------- Appendix --------------------

\clearpage
\appendix
\onehalfspacing

\renewcommand{\partname}{}
\renewcommand{\thepart}{}
\part{Appendix}
\localtableofcontents

% Reset page number and equation, float, and result numbers
\setcounter{page}{1}
\setcounter{lemma}{0}
\setcounter{corollary}{0}
\setcounter{proposition}{0}
\setcounter{equation}{0}

% Prepend new equation, float, and result numbers with "A" to show they originate in the Appendix
\renewcommand{\thelemma}{A.\arabic{lemma}}
\renewcommand{\thecorollary}{A.\arabic{corollary}}
\renewcommand{\theproposition}{A.\arabic{proposition}}
\renewcommand{\theequation}{A.\arabic{equation}}
\renewcommand{\thefigure}{A.\arabic{figure}}
\renewcommand{\thetable}{A.\arabic{table}}

\section{Proofs}

I might want to cite some papers that I didn't cite in the main text \citepapp{banks1990equilibrium}.
To help keep my word count low and my eventual copy editors happy, I don't want those citations showing up in the main text, so I'll create a separate bibiliography for the Appendix.
I just append \verb|app| to the ordinary citation commands.
For example, I made the citation in this paragraph with \verb|\citepapp{banks1990equilibrium}|.

\subsection{Proof of \autoref{res:maximizer}}

To prove the result, I'll rely on a couple of auxiliary lemmas.
These will be newly stated here, and their numbers will be prepended with ``A'' to show that they originate in the Appendix rather than the main text.
(I still put them in \verb|restatable| environments to save myself trouble in case I end up moving them into the body later, but this isn't necessary.)

\begin{restatable}{lemma}{resConcave}
  \label{res:concave}
  $U$ is strictly concave in $x$.
\end{restatable}

\begin{proof}
  We have
  \begin{equation}
    \label{eq:first-partial}
    \frac{\partial U(x, \alpha)}{\partial x} = -2 (x + 2 \alpha)
  \end{equation}
  and thus $\frac{\partial^2 U(x, \alpha)}{\partial x^2} = -2 < 0$.
\end{proof}

\begin{restatable}{lemma}{resCriticalPoint}
  \label{res:critical-point}
  $\frac{\partial U(x, \alpha)}{\partial x} = 0$ if and only if $x = -2 \alpha$.
\end{restatable}

\begin{proof}
  \autoref{eq:first-partial} implies that the condition $\frac{\partial U(x, \alpha)}{\partial x} = 0$ is equivalent to $x = -2 \alpha$.
\end{proof}

Now we can prove \autoref{res:maximizer}.
To restate the result, I'll use the command \verb|\resMaximizer*|, where \verb|resMaximizer*| is the command name I supplied to the second argument of the \verb|restatable| environment.

\resMaximizer*

\begin{proof}
  \autoref{res:concave} implies that first-order conditions are necessary and sufficient for maximization of $U$ with respect to $x$.
  The claim then follows from \autoref{res:critical-point}.
\end{proof}

\phantomsection
\addcontentsline{toc}{section}{Additional References}
\singlespacing
\bibliographystyleapp{apsr}
\bibliographyapp{references}

\end{document}
